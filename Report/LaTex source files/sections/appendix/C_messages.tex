\newpage
\section{Defined messages for the communication protocol}
\label{appendix:messages_list}

In the following appendix, every defined message is described in detail. As described in subsection \ref{subsec:communication} abount the communication protocol, each message is construct with a static header and variable body. The upcoming pages will illustrate how the body is built according to specific occurring events within the system.

\subsection*{(0001) Introduction}

\vspace{0.5cm}
\begin{bytefield}[endianness=little, bitwidth=2.4em]{5}
    \bitheader[lsb=16]{16-20} \\
    \bitbox{4}{0001}
    \bitbox{1}{EOT}
\end{bytefield}
\vspace{0.5cm}

\noindent
The message is sent from a client to the server upon connection. The SOURCE record is set to the Client identifier. When such identifier is yet unknown (before a user proceed with the log-in operation), the SOURCE record is set to 0000 by default. The server will associate the new connection (usually identified by its networking terms of IP address and telecommunication port) to the received identifier for future reference. If the message contained an empty SOURCE record (0000), such connection will be kept waiting until an identifier is assigned.

\subsection*{(0002) Close connection}

\vspace{0.5cm}
\begin{bytefield}[endianness=little, bitwidth=2.4em]{5}
    \bitheader[lsb=16]{16-20} \\
    \bitbox{4}{0002}
    \bitbox{1}{EOT}
\end{bytefield}
\vspace{0.5cm}

\noindent
The message is sent from a client to the server whenever the former quits its online status. Such event might be trigger by different reasons according to the client in use. The server closes any open socket with such client, freeing up some allocation space, and prepares to re-accept it during a future re-connection.

\subsection*{(0003)  Acknowledgment of the requested operation}

\vspace{0.5cm}
\begin{bytefield}[endianness=little, bitwidth=2.4em]{5}
    \bitheader[lsb=16]{16-20} \\
    \bitbox{4}{0003}
    \bitbox{1}{RS}
    \bitbox{1}{>0}
    \bitbox{1}{US}
    \bitbox{1}{P/N}
    \bitbox{1}{US}
    \bitbox{4}{optional error code}
    \bitbox{1}{EOT}
\end{bytefield}
\vspace{0.5cm}

\noindent
The message is sent from the server to a client whenever the latter requested an operation to be accomplished. It is constructed with up to two parameters: the former indicates whether the result of the operation was \textbf{P}ositive or \textbf{N}egative; on the other hand, the former is optionally added to a negative response in order to indicate extra details of the status. 

\medskip
\noindent
Optional error codes 
\begin{description}
    \item [0001] Combination (Toy, Parent) does not exist
\end{description}

\subsection*{(0004) Paring acknowledgment by the toy}

\vspace{0.5cm}
\begin{bytefield}[endianness=little, bitwidth=2.4em]{5}
    \bitheader[lsb=16]{16-20} \\
    \bitbox{4}{0003}
    \bitbox{1}{RS}
    \bitbox{1}{1}
    \bitbox{1}{US}
    \bitbox{4}{User Parent ID}
    \bitbox{1}{EOT}
\end{bytefield}
\vspace{0.5cm}

\noindent
The message is sent by the toy-client to the server upon accepting a previously received request of paring by a parent-client. It is composed of a single parameter which contains the unique identifier of such parent. The server takes care of updating the database of associated pairs, as well as sending an acknowledgement to the parent which requested the pairing.

\subsection*{(0005) Requester for Parent to enter the system (sign up/in)}

\vspace{0.5cm}
\begin{bytefield}[endianness=little, bitwidth=2.4em]{5}
    \bitheader[lsb=16]{16-20} \\
    \bitbox{4}{0005}
    \bitbox{1}{RS}
    \bitbox{1}{3}
    \bitbox{1}{US}
    \bitbox{1}{F}
    \bitbox{1}{US}
    \bitbox{3}{username}
    \bitbox{1}{US}
    \bitbox{3}{password}
    \bitbox{1}{EOT}
\end{bytefield}
\vspace{0.5cm}

\noindent
The message is sent by the parent-client to the server whenever the former requires to either sign-up or sign-in to the system. The message is constructed with three parameters: a flag indicating whether the operation is sign-in (0) or sign-up (1), the user-name (an email address) and a password (encrypted in SHA-254).

\subsection*{(0006) Response for Parent to enter the system (sign up/in)}

\vspace{0.5cm}
\begin{bytefield}[endianness=little, bitwidth=2.4em]{5}
    \bitheader[lsb=16]{16-20} \\
    \bitbox{4}{0006}
    \bitbox{1}{RS}
    \bitbox{1}{1}
    \bitbox{1}{US}
    \bitbox{4}{User Parent ID}
    \bitbox{1}{EOT}
\end{bytefield}
\vspace{0.5cm}

\noindent
The message is sent by the server to the parent-client in response to a previously received request of access (MSG\_ID equal to 0005). The only parameter is made of the assigned parent identifier that will be used by the client as SOURCE record for future messages.

\medskip
\noindent
\textbf{Attention:} if the parameter contains 0000, the access/registration in the system was refused by the server (possibly the user-name is either incorrect in the first case or already present for the second one).

\subsection*{(1001) Paring request to the toy}

\vspace{0.5cm}
\begin{bytefield}[endianness=little, bitwidth=2.4em]{5}
    \bitheader[lsb=16]{16-20} \\
    \bitbox{4}{1001}
    \bitbox{1}{EOT}
\end{bytefield}
\vspace{0.5cm}

\noindent
The message is sent by a parent-client to a toy-client in order to obtain "pairing privileges'. Without such privileges, the interactions between clients will be filtered by the server to prevent malicious behaviours. The request will be followed by a positive acknowledgment (see message with MSG\_ID equal to 0003) if the pairing is successful within a timeout limit. If such response is not retrieved before the timeout, the request is considered refused and needs to be resent again.

\subsection*{(2001) Playing session started by the parent}

\vspace{0.5cm}
\begin{bytefield}[endianness=little, bitwidth=2.4em]{5}
    \bitheader[lsb=16]{16-20} \\
    \bitbox{4}{2001}
    \bitbox{1}{EOT}
\end{bytefield}
\vspace{0.5cm}

\noindent
The message is sent by a parent-client to a paired toy-client in order to initiate a playing session with the latter. Then, the former client will wait for the toy to accept such playing session with a message which MSG\_ID is equal to 2002.

\subsection*{(2002) Playing session accepted by the Toy}

\vspace{0.5cm}
\begin{bytefield}[endianness=little, bitwidth=2.4em]{5}
    \bitheader[lsb=16]{16-20} \\
    \bitbox{4}{2002}
    \bitbox{1}{EOT}
\end{bytefield}
\vspace{0.5cm}

\noindent
The message is sent by the toy-client to the parent-client in response to a playing session request by the latter.

\subsection*{(2003) Interactive area activated}

\vspace{0.5cm}
\begin{bytefield}[endianness=little, bitwidth=2.4em]{5}
    \bitheader[lsb=16]{16-20} \\
    \bitbox{4}{2003}
    \bitbox{1}{RS}
    \bitbox{1}{PC}
    \bitbox{1}{US}
    \bitbox{1}{(n)}
    \bitbox{1}{EOT}
\end{bytefield}
\vspace{0.5cm}

\noindent
The message is sent by either a toy-client or a parent-client to the associated pair. It is used to indicate which area (LED on the toy, graphic on the Android app) is requested to turn ON during the interaction. The only parameter of the message is composed of an integer number between 1 and 8, which indicates the area of interest. For example, if the parent intends to turn ON the 7th LED of the plsuh toy, the parameter of the message will contains the number seven.

\subsection*{(2004) Interactive area de-activated}

\vspace{0.5cm}
\begin{bytefield}[endianness=little, bitwidth=2.4em]{5}
    \bitheader[lsb=16]{16-20} \\
    \bitbox{4}{2004}
    \bitbox{1}{RS}
    \bitbox{1}{1}
    \bitbox{1}{US}
    \bitbox{1}{(n)}
    \bitbox{1}{EOT}
\end{bytefield}
\vspace{0.5cm}

\noindent
The message is sent by either a toy-client or a parent-client to the associated pair. It is the reversed message for the previously described with MSG\_ID equal to 2003. Each number passed as parameter of the message will be turned off as either LEDs on the plush toy or graphical interface on the Android app. 

\subsection*{(2005) Playing session terminated}

\vspace{0.5cm}
\begin{bytefield}[endianness=little, bitwidth=2.4em]{5}
    \bitheader[lsb=16]{16-20} \\
    \bitbox{4}{2005}
    \bitbox{1}{EOT}
\end{bytefield}
\vspace{0.5cm}

\noindent
The message is sent by the parent-client to the toy-client whenever the former decides to quit the playing session. The plush toy, upon receiving the message, is able to reset its status and be prepared for a possible new interaction in the future.