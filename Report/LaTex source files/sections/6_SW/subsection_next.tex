\newpage
\subsection{Next steps for software}
\label{subsec:software_next} 

In conclusion of this section about the software engineering contribution to the developed project, a few final remarks need to be appointed. The previous pages illustrated a functioning prototype to test the smart plush toy. However, at this phase, many secondary aspects require to be fully explored in the future and we aim to present hereby the future "next steps" planned for the upcoming semester. 

\medskip
In the first place, the crucial goal of the upcoming months would be to enrich the \textit{Toygether system}, illustrated in the introduction (see section \ref{sec:fw}), from a so-called Local Area Network (LAN) to an extensive Wide Area Network (WAN). At the prototyping phase, every actor intercommunicating within the system are required to be connected to the same network location. For such reason, during the demos and tests accomplished this semester, the server would be identified with a local address (192.168.1.10). However, our initial aim with the project was to develop a connected device (i.e. the plush toy) that would communicate with their parent's smartphone whenever they are distant (at work for instance). The final goal will be to have a system that could be globally extended, allowing parents and children to stay connected whatever the geographical location (a parent could be in China while the rest of the family stays in Switzerland for example). To this intent, the upcoming semester will be focused on transporting the server application we developed to a public space on the web, reachable via a DNS from any connected client in the world. Moreover, this would require to instantiate an ever-running execution of such server application to be tested in this new situation.

\medskip
Lastly, we will explore the missing functions of the Android app. In fact, for a prototyping context, the major attention has been given to the interactive session between parents and children. This allowed the team to advance in every domain at a continuous pace. However, many "behind the scenes" aspects are still needed to fully develop and test. Such aspects could be the sign-up of the user, the first pairing with the toy and even the password recovery procedure. Once all those functions will be fully reached in the Android app, a possible iOS version could be envisioned for the future months of development of the software.
