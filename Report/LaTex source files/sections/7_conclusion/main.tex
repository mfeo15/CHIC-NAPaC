\newpage
\section{Conclusion}
\label{sec:conclusion}

% What a journey we had! For the past seven months, our team has kept working and improving every single day. We started from late October by not knowing each other at all. During the so-called "Ideation Week-end", we managed to foster our relationships and team NAPaC was born. Our team-name NAPaC (Not A Pasta Cooker) reflects both our Italian roots in the team (3 members out of six) and the tendency of never stopping brainstorming on possible ideas. We are a team which has always wanted to make the best out of this adventure. How could we ever settle then? One day, we eventually left behind us our first idea (a "Pasta Cooker"), but it would certainly not be the only example of not-product in our background.

% \medskip
% Days began to get shorter while the cold winter approached. This year, however, such signals were not only anticipating Christmas. The very first milestone arrived as well. Our first idea was a smart lamp you could use to virtually donate lighting to refugee camps, while yours was turned on. We really wanted to make an impact with this adventure. However, the lamp idea turned out to be the second not-product of our journey. We would not lie by saying that we did not fear to pivot at any time. We decided to completely re-position ourselves, while keeping the insights obtained so far. 

% \medskip The next milestone (a couple of months and university exams later) officially inaugurated our new domain: connecting parents and hospitalised kids with a smart plush toy. Pivoting from an idea (the "refugees lamp") to a new one (the "smart plush toy") does not imply restarting everything from the very beginning. On the contrary, we have expanded our vision without forgetting the main positive points we achieved so far. As already mentioned, we all loved engaging ourselves in an impactful project. Moreover, we have been growing a passion for "people connection" that our past project brought (connecting donators and refugees at distant), which perfectly matched our initial assigned topic of "Urban communities". We all appreciated working with a plush toy for kids. However, February could not yet be the month of our final solutions, as the journey was long ahead of us.

% \medskip 


%While a smart plush toy was a subject that appealed to our team, focusing on the hospitalisation issue felt to be too restrictive with this constraining context. In fact, the idea of a plush toy that can detect when a child cries to alert nurses has a good potential. However, the implementation requires a very high effort, due to the responsibilities in such a sensible context. Then, why not connecting children and parents, distant for any sort of reason and duration? We aimed at giving to young kids the power of communication, that new technologies (smartphones for example) brought. On the other hand, we also cared to give kids, already over-exposed to screens, an experience as less intrusive as possible in terms of technology. Could young children benefit from such long-distant and unintrusive connection with their parents?

While a smart plush toy was a subject that appealed to our team, focusing on a hospital context was too restrictive, given the sensitive nature of data we could be measuring. Then, why not connecting children and parents, distant for any kind of reason and duration? Since our final pivot, our goal has been to give to young children the power of distant communication, that new technologies such as smartphones could bring. However, we know that screens are far too present in most children's daily lives and thus wanted to design an experience as little intrusive as possible, in terms of technology. With Toygether, we succeeded in creating an "invisible" link from child to parents, which focuses on soft games and playful interactions.

\medskip The CHIC program has been a great opportunity for us to develop an interdisciplinary project from scratch, bringing together our best abilities in the fields of engineering, business and design. Five Milestones have already passed, which brought with them a huge number of unforgettable memories. There have been stressful days as well as enjoyable ones. We surely have enriched our academic year with an incredible experience that taught us many valuable skills, both in our domains of studies and, particularly, in the interdisciplinary context. %We have all joined this edition of CHIC on our own, but we have now grown to be an inseparable group (both in the NAPaC team and the rest of our peers as well).

%However, there is still a lot of work to do to bring a functional prototype to the world, which will happen with the fieldwork in China and the conclusion of our project in the Autumn semester.

\medskip We are proud to have finally found the perfect project to mark this experience: a smart plush toy to foster interaction between kids and parents at a distance. As we are writing these lines, we cannot help but recall the huge amount of work each of us had to put in to make this prototype come to life. % just do the sewing and you'll see what it's really like to bring a prototype to life! #proudmama

\medskip Although these last paragraphs sounded like a "goodbye", our adventure is far from being over. The CHIC program would not be named like that without the trip to China, coming in less than a month with new challenges. The final goal of our trip will be to get back to Switzerland with at least two fully working prototypes. One plush toy would be kept by Mr. Laperrouza, the founder of the CHIC adventure, who might showcase our prototype to the students of the following CHIC editions. The second plush toy would eventually stay with the \textit{Toygether} team, to present the project to investors or simply keep good memories from this very rich experience.
