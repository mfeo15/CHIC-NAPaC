\section{Related work}
\label{related-work}

% The literature review goes here.
% 4) Discuss studies about the impact on physical health
% 5) Discuss studies about the impact on academic performance
% 6) Discuss studies about the impact on mental health 

%%%%%%%%%%%%%%%%%%%%%%%%%%%%%%%%%%%%%%%%%%%%%%%%%%%%%%%%%%%%%%%%%%%%%%%%%%%%%%%%%%

According to the United Nations (\cite{united1998recommendations}) one can define migrant as “any person who changes his or her country of usual residence”. A stronger classification can be introduced on both time and space constraints (\cite{rossi2008impact}), depending whether the process is permanent, long-term (at least 12 months), short-term or seasonal for the first one and internal or international for the second. Purposes of recreation, holiday, visits to friends and relatives, business, medical treatment, or religious pilgrimages are, although, not considered as migration processes (\cite{united1998recommendations}). The effect of migrations on children, in particular, is highlighted under two scenarios correlated with the fact that the child follows his/her family during the process or it is left at home by either one or both parents (\cite{rossi2008impact}). Foster children that are not migrating with their parents, neither less still face direct exposure to mobility on their own, thus differentiating their experience from left-behind children (\cite{pilon2003foster}). In the Chinese context of LBCs the commonly accepted definition refers to children who stay at home when both of their parents or one parent relocates elsewhere to work for at least six month (\cite{lu2011left}). Thus the experienced phenomena in this situation is considered a to be a \textit{seasonal} (migrants spend at least 11 months away, but return home during the Spring Festival) and \textit{internal} (from rural to industrialized regions). The formal definition of children given in the \textit{Convention on the Rights of the Child} (CRC, Article 1 drafted by \cite{unicef1989convention}) states that "individuals below the age of 18" form the children population. On the other hand, it is fairly common to deal with resources following different conventions, being the 0-15 range the most adopted one due to medical (\textit{ex. women fertility}) or labor (\textit{ex. legal working age}) motivations in accordance with the research context.

%%%%%%%%%%%%%%%%%%%%%%%%%%%%%%%%%%%%%%%%%%%%%%%%%%%%%%%%%%%%%%%%%%%%%%%%%%%%%%%%%%

The study of \textcite{guo2017effect} reveals that the "overall effect" of parental migration on children’s health is uncertain. Several researches found positive effects of parental migration on children’s health (\cite{mundial2006development}; \cite{acosta2007impact}; \cite{anton2010impact}; \cite{stillman2012impact}) and have mainly attributed them for the increased incomes of the migrant-working parent. Remittances helped reducing childbirth mortality (\cite{hildebrandt2005effects}), increasing birth weight and decreasing the number of underweight newborns (\cite{frank2002other}). As a consequence, the development of chronic diseases slowed down (\cite{case2002economic}. Children started to be better nourished and the access to health services has been improved (\cite{nobles2006contribution}).

Other researches raised negative effects of parental  migration (\cite{amato1999nonresident}; \cite{kanaiaupuni2000reframing}; \cite{fernandez1998fathers}), mainly due to the reduction of parental care. \textcite{song2009health} found that left-behind children appeared to overuse the health services. A possible cause of this phenomenon is the attempt of replacing the lack of parental care by medical services. Another interesting fact from this study is that access to health care system depended on whom the children were left with. Statistically, maternal migration has a negative effect on children’s health. Several studies worldwide presented the maternal presence as essential for the development of a child (\cite{cortes2015feminization}; \cite{jampaklay2006parental}; \cite{macours2010seasonal}; \cite{thomas1994like}). With paternal migration, longer time and distance of migration positively affect children’s health, which would be a consequence of the increased remittances. Children health status reveals to be also affected by socio-economic factors, dependant on the household (\cite{behrman1996impact}). Furthermore, the need of initial capital to cover emigration costs could be part of the reason why poor parents tend not to migrate. It has been deduced that children’s health deteriorates with maternal migration but improves with longer distance and longer time of paternal migration. Increased income may have a positive effect while decreased parental care may have a negative effect. Lastly, neither a child’s age nor a child’s gender affects the health impact of parental migration. There are still problems, though, because the dependent variable of interest (health status) was self-rated (page 1154).

The study of Yu Bai and others (2016) rejected the hypothesis that parental migration negatively affects the academic performance of children (page 16). Interestingly, when one parent or both out-migrated, the academic performance of LBCs improved significantly. One possible reason is that the income effect of remittances is relatively large compared to the adverse effect of less parental care. The migrant households that experience rising incomes may be able to provide better nutrition, improve access to educational supplies and burden their children with less houseworks. In fact, the largest positive effects are found when both parents out-migrated (page 18). The heterogeneous analysis shows that the positive impact of parental migration on LBCs is greater for poor performing students. It may be the additional resources that are available to the households from newly out-migrating parents, that are able to overcome one or more of the educational barriers that were limiting the performance of the student. For example, when students are better nourished, their academic performance rises (shown by Lyo et al., 2012). Remittances might also be used to for other performance-enhancing investments such as remedial tutoring, additional books, learning software or associated computer hardware (page 20). The results of the heterogeneous analysis also demonstrated that the positive impact may be even larger when the parent who out-migrated had a low level of education. The results are consistent with the interpretation that there is a parental care-household resources trade-off when the parent of the student has the ability (from the level of education) to provide academic performance-enhancing care (from time spent tutoring the child). Therefore, the impact of parental migration on academic performance of LBCs differs based on the background of the parents, especially for the education levels of parents (shown by Sawyer, 2014) (page 21).

%%%%%%%%%%%%%%%%%%%%%%%%%%%%%%%%%%%%%%%%%%%%%%%%%%%%%%%%%%%%%%%%%%%%%%%%%%%%%%%%%%

%6) Discuss studies about the impact on mental health 

According to a study over 875 children (590 left-behind
children, 285 controls) in grades four–six in rural China, boys and girls whom parents migrated are at greater risk to develop depression compared to the control group (\cite{he2012depression}). The work extends a previous study (\cite{fan2010emotional}) conducted on the topic thus highlighting with different data and methodologies similar results. The emotional disturbances and the higher risks of depressions are particularly more evident in key age ranges as the one studied. During early adolescence years, changes in multiple domains including physical, sexual cognition
and social development occurs, leading to determinant consequences in the overall maturity process. Those illustrated psychological problems occurring in the late childhood and adolescence introduce useful indicators of risk for adult depression as a direct repercussion in the coming years of LBCs (\cite{kosterman2010assessment}).

The results presented in the Chinese context are further confirmed by a study deriving evidences from Romania (\cite{botezat2014impact}), where a similar phenomena of migration is growing from the country towards more occidental regions of Europe. The research align itself with the common agreement that migration processes lead to an overall increase of academic results as confirmed in other contexts such as Nicaragua (\cite{macours2010seasonal}), Philippines (\cite{arguillas2010impact}; \cite{yang2008international}) and China by appropriate studies. On the other hand, children whose at least one parent migrated for working purposes are more likely to express negative effects on their psychological well-being. The study illustrates substantial increases in depression attitudes as well as higher propensity to be bullied during the childhood, particularly in rural areas. Those insights match previous results from the literature on the mental health of left-behind children (\cite{gibson2011happens}; \cite{dreby2007children}; \cite{mazzucato2011transnational}), with prevalent consequences for those who are left to self take care of themselves and living in an environment without any adult overlook (\cite{lahaie2009work}).

On the other hand, according to a recent study on the topic (\cite{ren2016consequences}) different results from the one illustrated so far have emerged. The study of \textcite{ren2016consequences} was conducted on more than 3000 children aged 10-15 (comparable in terms of age to \textcite{he2012depression} study presented before). Although this latter highlights signs of depression risks as well, the conclusion is different from the rest of the literature. In consonance with the authors conclusion, in the Chinese context the family arrangements have little impact on the emotional well-being of children. The main texts of literature on the topic are either "covering" or "covered by" foreign countries, with a subsequently interpolation of results applied to the Chinese context. However, the context is extremely peculiar and can't be associated straightforwardly without imposing assumption that could bias the results. The conclusion of the study states that LBCs research in China could be considered a peculiar context due to the impressive number of internal migrants that is incomparable to any other country. This leads, between other reasons, to the fact that children face a common shared experience of being left-behind that would mitigate negative effects on their emotional well-being, particularly in some rural villages where LBCs count as the majority condition (\cite{hao2006discussion}).