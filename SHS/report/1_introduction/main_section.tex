\section{Introduction}
\label{sec:introduction}

Every year, Mainland China witnesses the biggest human mass migration process on Earth, during the Lunar New Year celebrations (\cite{zhang2015mass}). In February 2018, more than 385 million Chinese, according to \textcite{forbes2018springfestival}, converged back from cities to rural areas. This year, the purpose was to celebrate the year of the Dog, with their family (\cite{forbes2018springfestival}). Among those travellers, a substantial percentage consisted of migrant workers who moved from rural areas to more urbanized ones, in order to seek job opportunities. They live away from their family and usually all return home during the week before New Year, providing support to them via remittances throughout the year. According to \cite{guo2017effect}, "An increasing number of children in Chinese rural areas live far away from their parents, as about 80\% of the migrant workers leave their children behind in their home-village". For this reason, they are called "Left Behind Children"  (\begin{CJK*}{UTF8}{gbsn} 留守儿童\end{CJK*}, pinyin: \textit{liúshǒu'értóng}), because they are raised by other family members (the second parent figure or the grandparents) or, in the worst scenario, completely left on their own in their home village. The situation of departing parents has substantial impacts on the child's development, depending on multiple factors involved.

Our research study aims at understanding how communication technologies, connecting parents with their left behind children in rural China, could be beneficial. First, when focusing the study on a subset of the population, as we are doing by specifying "Rural China", a clear definition of this expression needs to be provided. According to \cite{shen1995rural}, the definition is strongly influenced by three crucial censuses, which occurred in 1982, 1987 and 1990. Those collected statistics have been used to divide home-towns of each Chinese province into either "agricultural" or rural areas, and "non-agricultural" or urbanized areas. The 1990-census is the most widely adopted allocation between "agricultural" and "non-agricultural" villages for such kind of researches. Therefore, our study will be based on this definition of "rural China". Nowadays, people are used to connecting with each other via always improving technologies (\cite{bird2018constantly}). We are thus interested in discovering patterns, in parallel to the enhancement of communication channels, between children and their distant parents. Our initial reasoning, biased by definition towards the western social context we have grown into, would indicate that the issue of left-behind children in China could harm both their physical and mental aspects. Moreover, our background suggests that a stronger communication between family members could lead to positive effects and eventually mitigate the issue. Such instinctive conclusions will be challenged during the course of the following research study. The Chinese context of left-behind children lays on the antipodes of our own and each conclusion we might consider trivial to draw needs to be reexamined in a different light. The study's goal is to explore if and how such context-issue located in China could benefit from enriched communication, via technological means.

To illustrate our work on the presented topic, we will undergo a series of key steps as follows: firstly, an overview in the form of literature review will showcase related work and previous investigations, regarding the left-behind children issue in China and the effects caused by it (section \ref{sec:related-work}); then, the methodologies followed during the research will be presented and covered (\ref{sec:methods}); obtained results of the research are thus introduced to the reader with the help of descriptive statistics (\ref{sec:results}); interesting patterns in the collected data will be retrieved and compared to the main issue regarding left-behind children, with the support of external expertise (\ref{sec:discussion}); and lastly, our final remarks and future scenarios on the presented work (\ref{sec:conclusion}).

