\section{Methods}
\label{sec:methods}

In this study, we seek to evaluate how communication technologies could alleviate the negative consequences of parental migration in rural China. As widely covered in the previous section, LBC are particularly subjective to retain long-term effects on mental well-being, due to the absence of parental care in crucial ages. Therefore, we aim to investigate how the increasing adoption of ICTs (\textit{Information and communications technologies}) might improve the condition of Chinese families in this context.

To this end, a research methodology has been structured as follows. The topic of ICTs for families will first be analyzed from a more global perspective, focusing on different geographical areas and situations. Our findings will be subsequently related to the local reality faced in rural China. In this manner, the importance of fostering communication technologies between distant parents and children will be explored, through a comparison of related studies within western countries (subsection \ref{methods-compare}). Then, after drawing subsequent assumptions, these latter will be tested through a study case realized in our geographical proximity (Switzerland, France and Germany). For this purpose, the recruitment of the samples will be highlighted (\ref{methods-recruitment}), before diving into the content of the survey (\ref{content-survey}).

\subsection{Comparative analysis on similar situations}
\label{methods-compare}

In this subsection, an analysis of different scenarios will be presented, investigating whether communication affects the mental well-being of children. The first sensible context that will be approached relates to hospitalized children, with long-term chronic diseases disrupting their previous social routine. The second scenario will explore the challenges faced by imprisoned parents to communicate with their children. The last scheme portrays transnational families, for which at least one of the parents left the family to migrate to another country. 

\vspace{4pt}
\paragraph{Hospitals}
In the research, conducted by \textcite{wadley2014exploring}, is shown that children undergoing long term hospitalization face psychological obstacles, due to a brutal change of previous daily habits. A long term isolation from their social life, both from their family and peers at school, causes concerns to their emotional well-being. Furthermore, this social dislocation occurs at a time when social engagement is crucial for children’s development, as explains the study of \textcite{hopkins2014embedding}. A comparable situation emerges between this context and the LBC issue, despite the different duration of separation. Furthermore, staying "out of sight" (\cite{yates2010keeping}, page 80) can lead to apprehension for the isolated child, about coming back to meet again after a long period of time. Therefore, ICTs' capabilities need to be further investigated. Innovative ICTs could not only be used to bridge distant people together, but also enhance emotional well-being by avoiding staying too long out of each other's mind. 

According to different studies (\cite{vernon1971psychological}, \cite{bossert1994stress} and \cite{thomson2012body}), hospitalization can increase children's stress. Parents, on the other hand, must continue working and may also experience anxiety and sense of guilt, not being able to be present for their child. This context, thus, shows similarities with migrating parents, leaving their children behind. As revealed in the studies of \textcite{stewart1998emotional} and \textcite{ryan2001happiness}, physical diseases could also result from emotional distress. Social support through ICTs, therefore, becomes essential for distant families. The research from \textcite{hopkins2013hospitalised} highlighted the impact of prolonged absence from school, possibly resulting in a child’s disengagement that can later cause employment difficulties. Challenges associated to unpredictability and vulnerability of children’s health within the hospital's context, presented by \textcite{yates2010keeping}, can be compared to fears and worries of migrating parents, regarding their LBC. 

Indirectly contributing to the emotional well-being of persons in sensitive settings, this first analysis of ICTs within hospitals already clearly demonstrates how technology could provide support and connection between LBC and out-migrating parents. 

\vspace{4pt}
\paragraph{Prison parenting}
% show how the missing/hard communication between parent and children affect the family
In 2007, 1.7 millions of children had an incarcerated parent in the United States of America (\cite{glaze2008parents}). \textcite{poehlmann2010children} state that these children experience a larger risk of substance abuse, behavioural problems and thus academic failure. Furthermore, poverty and changes in caregiver due to the parents' incarceration are either the cause or a "risk marker" (\cite{murray2008effects}) of children' problematics. 

Children's age is an important factor to take into consideration when approaching the parent-child interactions in this context. Infants, for instance, particularly need contact with their mother. In fact, \textcite{byrne2010intergenerational} showed that infants who stayed in the nursery program of the prison, where the mother resided, were more likely to keep a long term attachment with their mother (with respect to other children discharged from the program less than one year after their arrival). The actual parent-child mean of contact may differ with the age of the child since infants are not able to write letters or even have an oral conversation. Furthermore, when children are too young to  have their own opinion, caregivers might act as "gatekeepers" (\cite{enos2001mothering}). Thus, instead of facilitating the parental contact with the child, some might even limit it. Then, when becoming teenagers, children further develop their verbal skills and sometimes start visiting their incarcerated parents, hiding it from the caregivers (\cite{shlafer2010attachment}). An absence of contact with the parents might cause feelings of alienation to the child (\cite{shlafer2010attachment}) since no one ever gave enough affection and attention to let the child feel self-confident.

Parents, on the other hand, might feel stress from losing contact with their children (\cite{day2005incarcerated}, \cite{clarke2005fathering}, \cite{magaletta2001fathering}, \cite{roy2005gatekeeping}, \cite{arditti2008maternal}). Experienced symptoms of incarcerated parents are depression and anxiety (\cite{houck2002relationship}). The impressive outcome is that both direct and distal communications are able to improve the state of mind of incarcerated parents. \textcite{poehlmann2005incarcerated} showed that visits could decrease the depression of mothers in prison. Likewise, \textcite{loper2009parenting} discovered that mothers with more frequent phone calls, letters or emails were feeling less distress. As \textcite{casey2004children} associated joy and relief to visits, distant communications can also have a significant impact on parental mental health.

Regarding facility visitation policies, depending on the institutional security level, the parents are allowed to receive "full" contact visits (where physical contact is allowed), or "open" but without contact visits, or "barrier" visits (occurring with a Plexiglas barrier) (\cite{johnston1995parent}, \cite{sturges2005survey}). In some jails, visits can even happen only through a television transmission, where the incarcerated parent and the child are located in two separate areas. From this literature review, child visits can happen to be associated with emotional distress, together with uncomfortable environments and almost no opportunity for meaningful contact, as also explained by \textcite{arditti2003locked} and \textcite{loper2009parenting}.

Parents might prefer avoiding their child seeing them in a rather negative environment. Incarcerated parents thus prefer contacting their family with alternatives to visitations, such as written correspondence, because the settings are more under the parental control than on the authorities of the penitentiary. On the other hand, the child is less likely to overhear inappropriate discussions. The parent might be more relaxed and focused on the content of the interaction with the child, rather than on the environment. Visitation rooms, letters and phone calls become the child’s reality of the relationship with their parent. When children and parents did not manage to keep in contact, visits could also be stressful for the child. A perfect transition before initiating visitations could then be remote forms of contact. Alternatives for younger children, not yet apt to communicate orally, could make the use of a playful interaction. To conclude with regard to this context, distal communications offer flexibility (\cite{tuerk2006contact}), which is not given to incarcerated parents.

\vspace{4pt}
\paragraph{Transnational parenting}
%\subsubsection*{Transnational parenting}
The study of \textcite{bacigalupe2011virtualizing} provides another relevant example to test the importance, for growing children, of having communication channels with their family. The paper focuses on the increasing adoption of ICTs in migrant families and how technologies can ease the parenting role that became more difficult with the distance. 

%NOTE: this paragraph might even be put above as an introduction for both text. We might want to mix-up the two texts a little
The first key consideration the study highlights is the need to completely change people's state of mind about ICTs in family communications. The general public is often reminded of the negative impact an over-usage of technologies could bring in the family bonds. In fact, the adoption of technologies in a household might lead to unhealthy and, paradoxically, even distant connections with the other family members. However, when families face the situation of living at distance, the previous negative statements about ICTs need to be questioned. Therefore, a transnational family context cannot be modeled as the common family idea, where both parents and children can easily meet in reality.

An objective analysis of the study's content could not be possible without taking some distance with the paper. Face-to-face relationships are not anymore the only manner to maintain and build a social bond between people (\cite{bacigalupe2011virtualizing}). Introducing new possibilities of ICTs, this vision can be completely remodelled. \textcite{licoppe2004connected} describes this new era of communication technologies as "a continuous pattern of mediated interactions that combine into `connected relationships', in which the boundaries between absence and presence eventually get blurred" (page 135-136).

The research of \textcite{bacigalupe2011virtualizing} recalls that parental migration always existed since civilization exists. However, it often implied a sharp consequence on the household stability. The social concept of family often identifies the parental caregiving when growing children. Without ICTs, out-migrating parents would require frequent visits or, since recently, expensive international calls in order to keep a solid family bond. Most migrants, due to scarce resources, were not able to maintain such a functional long-distance family. This lack of parenting would eventually cause the irreversible problems during the child growth identified in the literature review (section \ref{sec:related-work}). Therefore, ICTs could bridge migrated parents with the rest of the family and provide the missing communication channel that leads to an unhealthy child's growth. We believe that, potentially, such new communication technologies could be the key to mitigate negative mental consequences, introduced in the context of LBC in China.

The role of ICTs is to allow families to engage in a "transnational care-giving". The rising adoptions of smart-phones, primary kind of ICT, is a clear signal of the importance such technologies have in the process. In fact, transnational families benefit from those tools to accomplish each small parental duty, necessary to the well functioning of the family such as bedtime routines or even hygiene check-ups. To further analyze this process, one needs to first define parental care-giving in detail. According to \textcite{finch1989family}, it could be expressed in five basic forms: 'financial' (i.e. remittances), 'practical' (i.e. sharing expertise), 'personal' (i.e. hands-on care of sickness), 'accommodation' (i.e. having a place to stay), and 'emotional' or moral support (\cite{bacigalupe2011virtualizing}). The adoption of communication technologies gives access for children to a new channel that effectively provides care-giving, as described, by migrant parents. Therefore, by using ICTs, transnational families are able to cover each of the forms of care-giving listed previously (\cite{baldassar2007transnational}). While some of these forms could be argued to need real proximity with the parents and in-person visits, all of them can still be handled at distance. Furthermore, with ICTs, the emotional and moral support from parental care-giving would be strongly enhanced and could, therefore, be mitigating the negative consequences of mental well-being for LBC.

In conclusion, the usage of ICTs in migrant families is less likely to have the negative impact that families living close together could face. ICTs actually provide a tool to strengthen family connections otherwise very difficult, if not impossible, to maintain. In this new light, we think that children of out-migrating parents could benefit from communication technologies to mitigate consequences identified in LBC mental health. By enriching accesses to different forms of parental care-giving at distance, as defined \textcite{finch1989family}, ICTs may be the key to improve the difficult conditions of migrant families.

\subsection{Recruitment}
\label{methods-recruitment}
In the analysis covered in the previous subsection, the importance of communication channels between growing children and the rest of the household has been clearly highlighted. Moreover, similar negative effects to the ones identified in the LBC's context were presented. Therefore, considered as not completely grown up yet, children below the legal age (18 years old in Europe, 15 years old in China) require strong communication with their parents to avoid negative mental consequences, regardless of the surrounding context. Even though each person's background is different, communication appeared to represent a key component of a healthy child's development.

In order to enrich our research, we decided to test this assumption in our own local context. The purpose of this section is to introduce the procedure that has been used, before presenting the results (see section \ref{sec:results}).

The objective of this phase is to understand the importance of communication in distant households and how ICTs might have helped mitigate negative effects related to that context. To this end, we decided to tackle the problem by identifying different situations present in our reality (a "Western context", primarily focused in Switzerland, France and Germany) in which households would be separated for a continuous amount of time. A complete recruited-population research has been done, which provided promising results for situations such as summer camps, scouting adventures or intercultural exchanges. Kids and teenagers whom joined such programs have spent from a couple of weeks (specifically the younger ones) up to six months or a year away from their family. As we previously did in the definition of LBC, by identifying constraints on both child age and family-separation time, the same must be done with the recruited population. While tackling the question from our local perspective, a trade-off aroused on the population characteristics. As mentioned, the younger the children and the less likely the children have spent more than a couple of weeks away from their family. On the other hand, increasing the age of the participants induced longer separation-times, better matching those of LBC (of at least six months). Therefore, during the discussion of the results, these factors shall be kept in mind to draw a more accurate conclusion.

In addition to children and teenagers recruitment, particular consideration from the parents' side has been taken. Since the study focused on the importance of communication, obtaining results concerning the two sides of the channel could potentially lead to richer conclusions. Moreover, especially when dealing with younger children, parents could provide more easily the needed information. 

While reading the previous paragraphs about the sample population recruitment phase, the reader might underline the impressive difference between the LBC issue and the distant-parenting experiences chosen. In fact, results collected from this selected population are difficult to relate to the Chinese issue because of the former experiences a family-detachment due to joyful and/or educative motifs. On the other hand, the migration decision is forced by much more difficult reasons. However, other possible populations have been considered to be recruited for such phase. Military recruits, who are often required to spend significant periods of time away from their home, or incarcerated parents who face the challenge of communicating with their family as already presented before could have led to richer results in the research. Although, due to difficulties with such recruitment, a trade-off has been achieved, thus leaving more accurate populations for further investigations and future iterations of the study.

To collect the data from those two described populations, an online survey has been prepared and posted on social media platforms. Initially, to be able to draw statistical conclusions, the expected minimum number of answers was set to 30 situations of distant parenting with children below 18 years old. Not being able to reach enough valuable data, we decided to turn ourselves to associations. The objective was not only to obtain more insights towards distant-parenting situations, but also to reach a more homogeneous population. Thus, we have been in contact with a consistent number of associations to spread out the survey (refer to annex \ref{appendix:contacts_list} for the list of contacts). The idea was to examine the question in our own region, by reaching out entities in the Canton de Vaud (eventually expanding then to Switzerland as a whole), whose work related to distant parenting experiences for children and their families. 

\newpage
\subsection{Content of the survey}
\label{content-survey}

As previously mentioned, the goal addressed with the survey was to determine the importance of maintaining a communication channel via ICTs, in any context of distant parenting. We aim to directly test on our European local reality how kids, away from their family, might find comfort in this connection. Even though the population we recruited and described in the previous section is mainly composed of families divided by joyful activities (intercultural exchanges, summer camps, etc.), homesickness can easily emerge. For this reason, even this type of joyful experiences can lead to a substantial decline in the emotional well-being, particularly for young children at their very first time away from home. The content of the survey intends to give an understanding about how essential channels of communication become for distant families in such context and, therefore, even more crucial in harder situations as the ones experienced by LBC in China.

The survey was written down in both French and English to maximize its accessibility. Post-processing tasks have then merged the results for comparisons. Moreover, the survey contained similar questions for both parents and children, but each from their own perspective to eventually later draw conclusions, with respect to both groups of family members.

To investigate the problem, the survey tried to collect enough demographic data of both children and parents, answering about their first experience of distant parenting. The idea is to, during the afterwards analysis, determine different patterns in how the communication channel might be correlated to gender, age or parental role (mother and father). This latter, for example, might be interesting because it could lead to results similar to the ones mentioned in our literature review for LBC in China. Different parental roles are key factors in how the distant parenting experience is handled. 
Furthermore, we investigated the kind of short-term "migration" to better understand how families have dealt with it, particularly in cases of long duration distant parenting and young ages of the children. 

Before diving into the analysis of the results, the reader may first want to explore the questions that have been asked in the survey (appendix \ref{appendix:survey_questions}). First, the context of the survey has been briefly explained. Then, questions were split into two categories, depending whether the person answering the survey has lived the situation of distant parenting as a child or a parent. Both child's and parent's gender were asked, in order to understand whether a stronger relationship could appear between distant mothers and children, as described in previously mentioned related work (section \ref{subsec:impact_physical_health}). Then, more precise questions about the context of distant parenting situations were asked (reasons, duration, age of the child). The focus of the survey was on the communication technologies used during that period of time (frequency, main purpose and mean of communication). Finally, the questioned people were asked how they would rate the importance of communicating with their distant relatives. Notice that the scale has been chosen from 1 to 7, with value 1 corresponding to insignificant and 7 to essential, in order to keep one neutral answer (4/7) and three possible choices of negative (3/7 and below) and positive answers (5/7 and above). Notice as well that when the importance of communication is presented as null (0/7), it means that no data has been received for a particular setting of distant parenting (i.e. in fig. \ref{fig:plot_8} with the frequency of communication corresponding to "once per month").

