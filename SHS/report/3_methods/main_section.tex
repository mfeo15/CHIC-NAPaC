\section{Methods}
\label{methods}

%Loneliness
%In this work we seek to ... (also with references on previous work)

In this study, we seek to evaluate how communication technologies could alleviate the negative consequences of parental migration in Rural China. As widely covered in the previous section, LBCs are particularly subjective to retain long-term effects on mental well-being, due to the absence of parental care in crucial ages. Therefore, we aim to investigate how the increasing adoption of ICTs (\textit{Information and communications technologies}) might improve the condition of Chinese families in this context.

To this end, a research methodology has been structured as following. The topic of ICTs for families will first be analyzed from a more global perspective, focusing on different geographical areas. Our findings will be subsequently related to the local reality faced in rural China. By this mean, we will be able to explore the importance of fostering communicative channels between distant parents and children, for children's growth, through a comparison of related studies within western countries (subsection \ref{methods-compare}). Then, our own study case testing our assumptions will be presented, from the recruiting of the samples (\ref{methods-recruitment}) to the content of the survey (\ref{content-survey}).

\subsection{Comparative analysis on similar situations}
\label{methods-compare}

In this subsection, an analysis of different scenarios will be presented, investigating whether communication affects the mental well-being of children. The first sensible context that will be approached relates to hospitalized children, with long term chronic diseases disrupting their previous social routine. The second scheme portrays transnational families, for which at least one of the parents left the family to migrate in another country. 

% Goal : show how communication is key for children's growth

% Analysis and related work with respect to long-distance parenting 

% Show the importance of communication between parents and children, for the child growth, in a long distance parenting 
% ex: hospitals, transnational families

%\subsubsection*{Hospitals}
\vspace{4pt}
\paragraph{Hospitals}
In the research, conducted by \textcite{wadley2014exploring}, is shown that children undergoing long term hospitalization face psychological obstacles, due to a brutal change of previous daily habits. A long term isolation from their social life, both from their family and peers at school, causes concerns to their emotional well-being. Furthermore, this social dislocation occurs at a time when social engagement is crucial for children’s development, as explains the study of \textcite{hopkins2014embedding}. Similarly, with LBCs, a comparison of contact rupture shapes, between hospitalized children and working parents on one case, and children and out-migrating parents on the other. Furthermore, staying "out of sight" (\cite{yates2010keeping}, page 80) can lead to apprehension of the isolated child, about coming back to meet again after a long period of time. Therefore, ICTs' capabilities need to be further investigated. Innovative ICTs could not only be used to bridge distant people together, but also enhance emotional well-being and avoid staying too long out of each other's mind. 

According to different studies (\cite{vernon1971psychological}, \cite{bossert1994stress} and \cite{thomson2012body}), hospitalization can increase children's stress. Parents, on the other hand, must continue working and may also experience anxiety and sense of guilt, not being able to be present for their child. This context, thus, shows similarities with migrating parents, leaving their children behind. As revealed in the studies of \textcite{stewart1998emotional} and \textcite{ryan2001happiness}, physical diseases could also result from emotional distress. Social support through ICTs becomes therefore essential for distant families. The research from \textcite{hopkins2013hospitalised} highlighted the impact of prolonged absence from school, possibly resulting in a child’s disengagement that can later cause employment difficulties. Challenges associated to unpredictability and vulnerability of children’s health within the hospitals context, presented by \textcite{yates2010keeping} , can be compared to fears and worries of migrating parents, regarding their LBCs. 

Indirectly contributing to the emotional well-being of persons in sensitive settings, this first analysis of ICTs within hospitals already clearly demonstrates how technology could provide support and connectedness between LBCs and out-migrating parents. 

\vspace{4pt}
\paragraph{Transnational parenting}
%\subsubsection*{Transnational parenting}
The study of \textcite{bacigalupe2011virtualizing} provides another relevant example to test the importance, for growing children, of having communication channels with their family. The paper focuses on the increasing adoption of ICTs in migrant families and how technologies can ease the parenting tasks that became more difficult with the distance. 

%NOTE: this paragraph might even be put above as an introduction for both text. We might want to mix-up the two texts a little
The first key consideration the study highlights is the need to completely change people's state of mind about ICTs in family communications. The general public is often reminded of the negative impact an over-usage of technologies could bring in the family bonds. In fact, the adoption of technologies in a household might lead to unhealthy and, paradoxically, even distant connections with the other family members. However, when parents face the situation of living in a different country than their children, the previous negative statements about ICTs for families need to be questioned. A transnational family context can therefore not be modeled as the common family idea, where both parents and children can easily meet in reality.

An objective analysis on the study's content could not be possible without taking some distance with the paper. Face-to-face relationships are not anymore the only manner to maintain and build a social bond between people (\cite{bacigalupe2011virtualizing}). With the introduction of new possibilities of ICTs, this vision can be completely remodelled. \textcite{licoppe2004connected} describes this new era of communication technologies as "a continuous pattern of mediated interactions that combine into `connected relationships', in which the boundaries between absence and presence eventually get blurred" (page 135-136).

The research of \textcite{bacigalupe2011virtualizing} recalls that parental migration always existed, since civilization itself was formed. However, it often implied a sharp consequence on the household stability. The social concept of family often identifies the parental care-giving when growing children. Without ICTs, out-migrating parents would require frequent visits or, since recently, expensive international calls in order to keep a solid family bond. Most migrants, due to scarce resources, were not able to maintain such a functional long-distance family. This lack of parenting would eventually cause the irreversible problems during the child growth identified in the literature review (section \ref{related-work}). Therefore, ICTs could bridge migrated parents with the rest of the family and provide the missing communication channel that leads to an unhealthy child's growth. We believe that, potentially, such new communication technologies could be the key to mitigate mental well-being consequences introduced in the context of LBCs in China.

The role of ICTs is to allow families to engage in a "transnational care-giving". The rising adoptions of smartphones, primary ICTs, is a clear signal of the importance such technologies have in the process. In fact, transnational families benefit from those tools to accomplish each small parental duty, necessary to the well functioning of the family such as bedtime routines or even hygiene check-ups. To further analyze this process, one needs to first define parental care-giving in detail. According to \textcite{finch1989family}, it could be expressed in five basic forms: 'financial' (i.e. remittances), 'practical' (i.e. sharing expertise), 'personal' (i.e. hand-on care of sickness), 'accommodation' (i.e. having a place to stay), and 'emotional' or moral support (\cite{bacigalupe2011virtualizing}). The adoption of communication technologies gives access for children to a new channel that effectively provides care-giving, as described, by migrant parents. Therefore, , by using ICTs, transnational families are able to cover each of the forms of care-giving listed previously (\cite{baldassar2007transnational}). While some of these forms could be argued to need real proximity with the parents and in-person visits, all of them can still be handled at distance. Furthermore, with ICTs, the emotional and moral support from parental care-giving would be strongly enhanced and could therefore be mitigating the negative consequences of mental well-being for LBCs.

In conclusion, the usage of ICTs in migrant families is less likely to have the negative impact that families living close together could face. ICTs actually provide a tool to strengthen family connections otherwise very difficult, if not impossible, to maintain. In this new light, we think that children of out-migrating parents could benefit from communication technologies to mitigate consequences identified in LBCs mental health. By enriching accesses to different forms of parental care-giving at distance, as defined \textcite{finch1989family}, ICTs may be the key to improve the difficult conditions of migrant families.

\subsection{Recruitment}
\label{methods-recruitment}
In the analysis covered during the previous subsection, the importance of communication channels between growing children and the rest of the household has been clearly highlighted. Moreover, similar negative effects to the ones identified in the LBCs' context were presented. Therefore, for children below the legal age (18 years old in Europe, 15 years old in China), considered as not completely grown up yet, require communication with their parents to avoid mental well-being consequences, regardless of the surrounding context (long-term hospitalization or out-migrating family members). Even though each person undergoes different difficulties through particular conditions, communication seemed to already clearly represents a key component of healthy children's growth.

In order to enrich our research, we decided to test this assumption in our own local context. The purpose of this section is to introduce the procedure that has been used, before presenting the results in the following section.

The objective of this phase is to understand the importance of communication in distant households and how ICTs might have helped mitigate negative effects related to that context. To this end, we decided to tackle the problem by identifying different situations present in our reality (a "Western context", focused on evolution in Switzerland, France, Italy and Germany) in which households would be separated for a continuous amount of time. A complete recruited-population research has been done, which provided promising results situations such as summer camps, boy-scouting adventures or intercultural exchanges. Kids and teenagers whom joined such programs have spent from a couple of weeks (specifically the younger ones) up to six months or a year away from their family. As we previously did in the definition of LBCs, by identifying constraints on both child age and family-separation time, the same must be done with the recruited population. While tackling the quest from our local perspective, a trade-off aroused on the population characteristics. As mentioned, the younger the children and the less likely the children have spent more than a couple of weeks away from the family. On the other hand, by increasing the age of the participants rise also the separation-times, better matching those of LBCs at least six months without one or both of their parents. Therefore, during the discussion of the results, these factors shall be kept in mind to draw a more accurate conclusion.

In addition to children and teenagers recruitment, particular consideration for the parents side has been taken. Because of the study focused on the importance of communication, obtaining results concerning the two sides of the channel would be useful and could potentially lead to a more interesting conclusion. Moreover, especially when dealing with younger children, parents could provide more easily the needed information. 

To collect the data from those two described populations, an online survey has been prepared and posted on social media platforms. Initially, to be able to draw statistical conclusions, the expected minimal number of answers was set to 30 situations of distant parenting with children below 18 years old. Not being able to reach enough valuable data, we decided to turn ourselves to associations. The objective was not only to obtain more insights towards distant-parenting situations, but also to reach a more homogeneous population. Thus, we have been in contact with a consistent number of associations to spread out the survey (refer to annex \ref{appendix:contacts_list} for the list of contacts). The idea was to examine the question in our own region, by reaching out entities in the Canton de Vaud (eventually expanding then to Switzerland), whose work related to distant parenting experiences for children and their families. 
% Vogliamo mettere un Annex qui? Magari indicare per bene i nomi che abbiamo contattato e che ci hanno risposto? Cosi abbiamo una reference seria di chi sono questi dati che ci provengono. Specialmente se iniziamo ad avere i risultati della signora di Ginevra, può essere serio


\subsection{Content of the survey}
\label{content-survey}

As previously mentioned, the goal addressed with the survey is to determine the importance of maintaining a communication channel via ICTs, in any context of distant parenting. We aim to directly test on our European local reality how kids, away from their family, might find comfort in this connection. Even though the population we recruited and described in the previous section is mainly composed of families divided by joyful activities (intercultural exchanges, summer camps, etc.), homesickness can easily emerge. For this reason, even this type of joyful experiences can lead to a substantial decline in the emotional well-being, particularly for young children at their very first time away from home. The content of the survey intents to give an understanding about how essential channels of communication become for distant families in such context and therefore even more crucial in harder situation as the ones experienced by LBCs in China.

The survey was written down in both French and English to maximize its accessibility. Post-processing tasks have then merged the results for comparisons. Moreover, the survey contained similar questions for both parents and children, but each from their own perspective to eventually later draw conclusions, with respect to both groups of family members.

In order to investigate the problem, the survey tried to collect enough demographic data of both children and parents answering. The idea is to, during the afterwards analysis, determine different patterns in how the communication channel might be more or less prominent according to sex, age or parental role (mother and father). This latter, for example, might be interesting because it could lead to results similar to the ones mentioned in our literature review for LBCs in China. Different parental roles are in fact key factors in how the distant parenting experience is handled. 
Furthermore, we investigated the kind of short-term "migration" to better understand how families have dealt with it, particularly in cases of long duration distant parenting and young ages of the children.

Before diving into the analysis of the results, the reader may first want to explore the questions that have been asked in the survey (appendix \ref{appendix:survey_questions}). First, the context of the survey has been briefly explained. Then, questions were split in two categories, depending whether the person answering the survey has lived the situation of distant parenting as a child or a parent. Both child's and parent's gender were asked, in order to understand whether a stronger relationship could appear between distant mothers and children, as described in previously mentioned related work (section \ref{subsec:impact_physical_health}). Then, more precise questions about the context of distant parenting situations were asked (reasons, duration, age of the child). Last but not least, the focus of the survey was on the communication technologies used during that period of time (frequency, main purpose and mean of communication). Finally, the questioned people were asked how they would rate the importance of communicating with their distant relatives. Notice that the scale has been chosen from 1 to 7, 1 corresponding to insignificant and 7 to essential, in order to keep one neutral answer (4/7) and three possible choices of negative (3/7 and below) and positive answers (5/7 and above). When no answer was received, the null grade (0/7) has been used.

%Explain that rates of importance of communication have beeen given on a scale from 0 to 7, 0 corresponding to insignificant and 7 to essential [and not from 0 to 10] -> 3 negative answers, 3 positive answers, and one can be neutral.