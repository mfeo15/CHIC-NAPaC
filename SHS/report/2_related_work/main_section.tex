\section{Related work}
\label{related-work}

A large amount of studies have been conducted about the reality of left-behind children, on different localities around the world. These studies have been focusing on different impacts of parental migration, from the children's perspective. Interestingly, this phenomenon is not only restricted to China (\cite{song2009health}; \cite{he2012depression}; \cite{guo2017effect}; \cite{fan2010emotional}; \cite{bai2017effect}), but also frequently appears in Mexico (\cite{sawyer2016money}; \cite{kanaiaupuni2000reframing}; \cite{hildebrandt2005effects}; \cite{fernandez1998fathers}; \cite{dreby2007children}), Latin America (\cite{mundial2006development}; \cite{acosta2007impact}; \cite{anton2010impact}), Romania (\cite{botezat2014impact}), Philippines (\cite{yang2008international}; \cite{cortes2015feminization}; \cite{arguillas2010impact}) and Thailand (\cite{jampaklay2006parental}). In this section, specific terms used throughout the research will be defined. Then, related work will be reviewed, about the impact of parental migration on physical health, academic performance and mental health of left-behind children. 

\subsection{Definitions and terminology}
\label{subsec:definitions_and_terminology}

According to the United Nations (\cite{united1998recommendations}), the commonly accepted definition of 'migrant' is "any person who changes his or her country of usual residence". A more precise classification can be introduced with time and space constraints (\cite{rossi2008impact}). The process is then defined as "permanent", "long-term" (at least 12 months) or "seasonal", with either an internal or an international migration. Purposes of recreation, holiday, visits to friends, business, medical treatment, or religious pilgrimages are, although, not considered as migration processes (\cite{united1998recommendations}). Two scenarios of parental migration can appear. Either the parents migrate with their children, or they leave them behind (\cite{rossi2008impact}). Another possible result of parental migration is related to the issue of foster children. Although they share the same loss of parental care, the complete interruption of communication with both parents determines a completely different case that cannot be discussed nor compared with the two previously mentioned scenarios (\cite{pilon2003foster}). The commonly accepted definition of LBCs (\textit{left-behind children}) refers to children who stay at home when one or both parents relocate elsewhere, to join the labor force, for at least six months (\cite{lu2011left}). In the Chinese context, parental migration is considered to be seasonal and internal, since migrants move from rural to industrialized regions, spending yearly around 11 months away and returning home during the \textit{Spring Festival}. The formal definition of 'children', given in the CRC (\textit{Convention on the Rights of the Child}), Article 1 drafted by \textcite{unicef1989convention}, states that "individuals below the age of 18" form the children population. However, other conventions to define 'children' can be found. The 0-15 age range is the most adopted one, in accordance with the research context, due to medical (women fertility) and labor (legal working age) reasons.

\subsection{Impact on LBCs' physical health}
\label{subsec:impact_physical_health}

The study of \textcite{guo2017effect} reveals that the "overall effect" of parental migration on children's health is uncertain. 

Several researches found positive effects of parental migration on children's health (\cite{mundial2006development}; \cite{acosta2007impact}; \cite{anton2010impact}; \cite{stillman2012impact}) and have mainly attributed them for the increased incomes of the family. Remittances helped reducing childbirth mortality (\cite{hildebrandt2005effects}), increasing birth weight and decreasing the number of underweight newborns (\cite{frank2002other}). Children are better nourished and can take advantage from an eased access to health services (\cite{nobles2006contribution}). As a consequence, higher incomes enable the children to better manage chronic health problems through medication (\cite{case2002economic}).

Other researches raised negative effects of parental migration (\cite{amato1999nonresident}; \cite{kanaiaupuni2000reframing}; \cite{fernandez1998fathers}), mainly due to the reduction of parental care. \textcite{song2009health} found that LBCs started to overuse health services. A possible cause of this behaviour is the attempt to replace the lacking parental care by medical services. Another intriguing fact from this study is that the access to health services depended on whom the children were left with. Statistically, maternal migration had a negative effect on children's health. This result confirms several studies showing that maternal presence is essential for the development of a child (\cite{cortes2015feminization}; \cite{jampaklay2006parental}; \cite{macours2010seasonal}; \cite{thomas1994like}). With paternal migration, longer time and distance of migration positively affected children's health, again as a consequence of increased remittances. Children's health status revealed to be also affected by socio-economic factors, dependant on the household (\cite{behrman1996impact}). One reason why poor parents do not out-migrate could be the need of initial capital to cover emigration costs, for transportation and accommodation reasons. Lastly, health impact of parental migration has been shown to be independent to the child's age and gender (\cite{guo2017effect}). 

\subsection{Impact on LBCs' academic performance}
\label{subsec:impact_academic_performance}

The study of \textcite{bai2017effect} rejected the hypothesis that parental migration deteriorated the academic performance of children. Additionally, they found that when one parent or both out-migrated, the academic performance of LBCs was even significantly improving, compared to the rest of the students. Increased incomes can therefore bring a larger impact than the reduction of parental care. Rising incomes might provide several benefits for LBCs, like a better nutrition, improved access to educational materials and less housework responsibilities. An analysis about the heterogeneity of the impact, among children of different households, showed that the positive effect of parental migration was even larger for students initially poorly performing at school. An explanation for LBCs' academic improvements is the availability of additional materials, which might have led them overcoming educational barriers that limited their performances. For instance, better nourished students see their academic performance rising (\cite{luo2012nutrition}). Remittances could also be invested in tutoring and other additional learning materials like books, computers and learning software. The positive impact of parental migration has been found to become larger when the out-migrating parent had a poor level of education. If the parent has a high level of education, a trade off between parental tutoring and migrant remittances appears. Therefore, depending on the out-migrating parent's background, the impact on the academic performance can significantly differ among LBCs (\cite{sawyer2016money}). 

\subsection{Impact on LBCs' mental health}
\label{subsec:impact_mental_health}

According to the study of \textcite{he2012depression}, LBCs were at greater risk to develop depression, compared to the other students. The work confirmed a previous study from \textcite{fan2010emotional}, conducted on the same topic. The emotional disturbances and higher risks of depression are more evident in specific age-ranges. During the early adolescence, changes in sexual cognition and social development occur, leading to determinant consequences for the overall maturity process. Without parental care, psychological problems occurring in the late childhood and adolescence can introduce direct repercussions on LBCs' mental health, increasing the indicators of adult depression risks  (\cite{kosterman2010assessment}).

The results presented in the Chinese context were further confirmed by another study, deriving evidences from Romania (\cite{botezat2014impact}), where the migration fluxes end towards more occidental European regions. The study illustrates substantial increases in depression, as well as higher propensity to be bullied during the childhood, more particularly in rural areas. These research insights match previous results from the literature on the mental health of left-behind children (\cite{gibson2011happens}; \cite{dreby2007children}; \cite{mazzucato2011transnational}). Children left alone to take care of themselves, living in an environment without any adult supervision, result in even larger risks of depression (\cite{lahaie2009work}).

\textcite{ren2016consequences} realized a more philosophical research about the possibility of wrongly perceiving causes of depression with Chinese LBCs. The research compared the difference of family "arrangements", or emotional bonds between parents and children, in China and other countries where the phenomenon of parental migration frequently occurs (i.e. Nicaragua, Philippines, Mexico). They concluded that, in the Chinese context, these family arrangements have "little impact on the emotional well-being of children". For this reason, researches about LBCs in the specific context of China should be considered a singularity of its own, due to the impressive number of internal migrants, incomparable to any other country. From the Chinese point of view, since the situation of leaving children behind became more common in rural areas (\cite{hao2006discussion}), non-Chinese researchers shall be cautious about their belief of understanding the local social context in China.



