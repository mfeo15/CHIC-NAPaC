\section{Conclusion}
\label{sec:conclusion}

Initially, the aim of this research was to understand how communication technologies could enhance the condition of left-behind children in rural China. 

For this purpose, a first step was to investigate the related work about left-behind children, not only in China but all around the globe, to compare the eventual differences among families of different cultures living the same situation of distant parenting. The following conclusions are applicable to any family at distance, independently from the family origins. The "overall effect" (\cite{guo2017effect}) of parental migration on children's physical health is unclear because several positive and negative effects might add themselves. The main cause of children's physical health improvements comes from the increase of family incomes, due to the parental remittances. In fact, birth weight has been shown to increase because children are better nourished, thus decreasing childbirth mortality. On the other hand, the main cause of negative impact on the children's physical health is related to the lack of parental care. Concerning the academic performance of left-behind children, their capabilities have been shown to be significantly improved, again thanks to the rising family incomes. Additional educational materials and better nutrition provided significant benefits to help the children better perform at school. However, from the mental point of view, left-behind children seem to present larger probabilities of developing depression, related to the lack of parental care. However, this result might not be applicable in the Chinese context, due to the different family settings between parents and children, more physically distant compared to Western families. 

Then, our research, properly speaking, started by seeking comparable situations of left-behind children in our geographical proximity. Three scenarios have been depicted and related work on each of those have been reviewed. The first one related to children at distance from their parents because of hospitalization. The main difference with this scenario and Chinese left-behind children is that the rupture in the routine of the hospitalized children might be of stronger negative impact, relative to the emotional bonds between European parents and children, and to the habit of living distant from each other. in this setting, communication technologies revealed to be helpful in supporting the children from such an intense change in their daily habits. The second scenario was focusing on the situation of prison parenting. In this case, the main difference with the Chinese context is that incarcerated parents are forbidden to be present for the children. In the Chinese context, parents chose to live this situation to enhance their children situation. The situation of distant parenting creates thus an aspect of emotional distress on the parental side, which is much stronger than in the Chinese context. Chinese parents might be even more respected by their children for everything they do in order to enhance their possibilities. The major positive aspect of communication technologies for incarcerated parents was to bring flexibility and control over the communication, which appeared to be very difficult in this type of situation. The third and last explored scenario was about transnational parenting in general. This situation appeared to be the most similar one to the condition of Chinese parents, also out-migrating, within the country (opposed to the transnational example), to seek for better job opportunities. The example of transnational parenting seems to give a very promising future to communication technologies in the Chinese context. In fact, parents at distance from their children use these available communication channels to engage and keep their caregiver duties towards their children, ensuring they have everything needed. 

In all three scenarios, communication technologies demonstrated to provide support to distant parents and children. Since we could not test these deductions on the Chinese reality, we wanted to make sure that these assumptions were applicable in our own locality. For this reason, we first sought to contact prisoners, then families with military parents, unfortunately in vain, before ending up with children having spent time away from their parents for more joyful experiences, such as summer camps, scouting adventures and intercultural exchange programs. Nevertheless, the reasoning was that if the importance of communication technologies is key in such positive and enjoyable settings, communication technologies could be considered even more essential in a more difficult configuration such as the one of interest in the Chinese context. Different associations have been contacted and have accepted to forward our survey to the children (and their parents when possible) having participated in their programs. The main goal of this survey was to understand how useful and appreciated were the communication technologies, from both children's and parents' points of view. An objective representation of the results, shown with statistical box-plots, clearly demonstrated how important communication technologies  are for distant parents and children, in such fortunate settings. A more in-depth analysis revealed that daughters judged more important communication technologies than did male children. 

To understand whether these deductions could be applicable in the Chinese reality of left-behind children, as we could not interview Chinese children, students or parents having lived the situation of left-behind children, we managed to get an interview with Ms. Graezer Bideau, expert in Chinese anthropology and urban sociology. This expertise has let us validate how essential is the communication between distant parents with their left-behind children, for who they do all these sacrifices. 

As an output of our research, an interview protocol has been realized. This tool can be helpful for any later research on the field, around the question of communication technologies for left-behind children and parents at distance. 

To conclude, an interesting question could be to reflect on how communication technologies would, in the end, impact the Chinese habits. Considering this more physically distant relationship that parents have with their children, how would these communication technologies perturb the family arrangements present nowadays?
