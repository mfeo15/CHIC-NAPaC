\section{Methods}
\label{methods}

%Loneliness
%In this work we seek to ... (also with references on previous work)

In this study we seek to evaluate how communication technologies could alleviate the negative consequences of parental migration in Rural China. As widely covered in the previous section, LBCs are particularly subjective to retain long-term effects on mental well-being due to the absence of parental care in crucial ages. We aim to investigate how the increasing adoption of ICTs (\textit{Information and communications technology}) might drastically improve the condition Chinese families in this context, by radically ...

To this end, we will structure a research methodology by analyzing the topic from global perspective at first and subsequently relate our findings to the local reality of interest, rural China. By this mean, we will be able to explore the importance for child growth of enhancing communicative channels with their parents through an evaluation and comparison of related studies in western countries (subsection \ref{methods-compare}); it will be followed by the definition of a similar study case and recruiting of samples to test our assumptions in our context (\ref{methods-recruitment}); and finally ... (\ref{methods-interviews}).

\subsection{Comparison with Western countries}
\label{methods-compare}
Analysis and related work with respect to long-distance parenting 

Show the importance of communication between parents and children, for the child growth, in a long distance parenting 
ex: hospitals, transnational families

\subsection{Recruitment}
\label{methods-recruitment}
We recruited participants by ...

\subsection{Interviews}
\label{methods-interviews}
Interviews were conducted in August 2016 ...